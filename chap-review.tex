\chapter{Citations and Bibliography}\label{chap:review}

This chapter should have been a survey on the history of \TeX{} and \LaTeX{}, and a comparison to conventional word processors in preparing academic documents.  Due to lack of time on the author's part, and also the abundance of such discussions on the web, we look at ways to prepare the bibliography and citations instead.

\section{The \texttt{*.bib} File}
First of all, bear in mind that your bibliography file (\verb|*.bib| files) is like a database like Mendeley.  That means you can maintain a centralised list, and reuse it for all your publications.  \LaTeX{} will only list sources that you actually cite in the text for each document, according to the bibliography and citation style you select in each document.  But you can still hack it so that your own publications are listed, even if you did not cite it.
 

\begin{figure}[htb!]
\begin{lstlisting}[language={}]
@BOOK{latex:companion,
  title = {The {\LaTeX} Companion},
  publisher = {Addison-Wesley},
  year = {2004},
  author = {Frank Mittelbach and Michel Goosens and Johannes Braams and David Carlisle and Chris Rowley},
  series = {Addison-Wesley Series on Tools and Techniques for Computer Typesetting},
  address = {Boston, MA, USA},
  edition = {2nd}
}
\end{lstlisting}
\caption{A BibTeX Entry}\label{fig:bibtex}
\end{figure}

As an example, in \verb|mybib.bib| I created a Bib\TeX{} entry with JabRef, the source text of which is shown in Figure~\ref{fig:bibtex}.

One thing to note about authors' names: Bib\TeX{} recognises ``Mittelbach'' as the last name for both \texttt{Frank Mittelbach} and \texttt{Mittelbach, Frank}.  So for a name like ``Mohd Hanafi Mat Som'' and ``Lim Lian Tze'', you would have to specify it as \texttt{Mat Som, Mohd Hanafi} and either \texttt{Lian Tze Lim} or \texttt{Lim, Lian Tze} for Bib\TeX{} to recognise ``Mat Som'' and ``Lim'' as the last name correctly.  In addition, note that my surname (or family name) consists of multiple words, thus enclose it with braces to avoid surprises, like so: \texttt{Mohd Hanafi \{Mat Som\}}.

\section{Citations using the \texttt{apacite} package}
The \verb|unimapcgs| class imports the \verb|apacite| package which provides citation mechanism as per required by the CGS, so see its documentation for more details.  On a MiK\TeX{} installation,
% it should be in \url{texmf/doc/latex/natbib/natbib.dvi} or \url{natbib.pdf}, in the path where you installed MiKTeX. 
use the command prompt to issue \lstinline|mthelp --view apacite| to access the documentation.
On TeXLive, simply type \verb|texdoc apacite| and the documentation will be displayed automatically, if it's found on your machine.

The basic citation commands are \verb|\citet| and \verb|\citep|, which stands for \emph{textual} and \emph{parenthetical} citation respectively.  They take extra arguments, too, for adding notes in the citations.  Please refer to the \verb|apacite| manual. 

\subsection{Author-Year System}
The default bibliography style is APA:

\begin{itemize}[nosep]
\item \verb|\citet{latex:companion}| $\to$ \citet{latex:companion}
\item \verb|\citet[chap.~2]{latex:companion}| $\to$ \citet[chap.~2]{latex:companion}
\item \verb|\citep{latex:companion}| $\to$ \citep{latex:companion}
\item \verb|\citep[chap.~2]{latex:companion}| $\to$ \citep[chap.~2]{latex:companion}
\item \verb|\citep[see also][]{latex:companion}| $\to$ \citep[see also][]{latex:companion}
\item \verb|\citep[see also][chap.~2]{latex:companion}| $\to$ \citep[see also][chap.~2]{latex:companion}
\item \verb|\citet{latex:companion,roberts}| $\to$ \citet{latex:companion,roberts}
\item \verb|\citep{latex:companion,roberts}| $\to$ \citep{latex:companion,roberts}
\end{itemize}

You may also want to write only the author's name or year occassionally:

\begin{itemize}[nosep]
\item \verb|\citeauthor{latex:companion}| $\to$ \citeauthor{latex:companion}
\item \verb|\citeyear{latex:companion}| $\to$ \citeyear{latex:companion}
\item \verb|\citeyearpar{latex:companion}| $\to$ \citeyearpar{latex:companion}
\end{itemize}
 

\subsubsection{Testing the TOC}

This section is a dummy section. But please avoid this subsubsection in your thesis. 